%###################################################################################
%	File: 					part_Theory.tex
%
%	Author: 				Christoph Müller
%	Date started writing:	13.07.2015	
%
%	Topic: 					part of Master Thesis 
%	Title: 					"Fast MRI methods for metabolite mapping."
%	Section:				"Theory"
%
%###################################################################################

\section{Theory}



\subsection{MRI}
The clinical magnetic resonance imaging makes use of the spin property of hydrogen nuclei. Hydrogen has one proton in its nucleus and one electron surrounding the nucleus. In this work we will only matter about atom nuclei and their physical properties.

\subsubsection{The Spin}
The hydrogen nucleus has four important physical properties: \textit{mass}, \textit{electric charge}, \textit{magnetism} and \textit{spin}. 
\textit{Mass} is a property of bulk matter, which is generated by gravitational fields, which forces very big physical objects to attract each other. However this force of attraction at the size of nuclei is very small and in most cases overlaid by another attractive force induced by the \textit{electrical charge}. Strong electrostatic interactions between positive charged nuclei and negative charged electrons bind all atoms and molecules for example. The \textit{magnetism} and \textit{spin} properties have almost no effect on normal chemical or physical behavior in our macroscopic world, but give scientists an magnificent tool to spy into microscopic and internal structures of objects.
The spin is a rather complex property of quantum mechanical nature. In a very classical view one can say that the proton has a intrinsic rotation, which causes the nucleus to have an angular momentum. Together with the electric charge the rotation induces a magnetic dipole moment:
\begin{equation}
\vec{\mu} = \gamma \vec{I}
\end{equation}
$\vec{I}$ is here the nuclear spin and $\gamma$ a element specific proportionality factor. This magnetic moment behaves like a tiny magnet, what means it can be influenced by magnetic fields or electromagnetic waves. 




\subsection{$^{13}$C-MRI}
\subsection{Chemical Shift}

and so on